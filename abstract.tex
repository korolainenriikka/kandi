%abstract should be max 100 words

% 1st paragraph: summarized: what is this thesis about, motivation also
A common problem with creating successful large-scale machine learning systems is the fact that significant effort is needed to keep the system working over long periods of time. This is mainly caused by unpreparedness to changing environment causing changes in the data source, also known as concept drifts. In order to adapt to these drifts, advanced and case-tailored maintenance schemes are needed.

% 2nd paragraph prepares readers to follow
This thesis searches for efficient model updating schemes for the specific case of bringing machine learning methods to the maritime domain, to applications such as traffic forecasting. Given the model type, recurrent deep neural networks, and case-specific requirements, such as large data and comparatively loose end-to-end latency requirements, the optimal drift-coping methods are searched for.

% 3rd paragraph: results and implication (conclusions summary essentially)
Based on literature it is found that utilizing data and model monitoring, optimized neural network training using data parallelism and network approximation, combined with  identifier-based data versioning would most likely enable consistent accuracy of the models despite environment changes. However, it is found that for many approaches the usability for the specific case cannot be determined without experimental benchmarking of several options.  %add here: more motivation.

%some abstract drafting... (what to include)
% essence of your paper
% prepares readers to follow
%  remember key points
  
%  the context - maritime
%  the general topic under study - big data processing systems
%  the specific topic of your research - updating neural networks efficiently
%  the central questions or statement of the problem your research addresses - which updating workflow is optimal for facilitating model maintenance
%  what’s already known about this question, what previous research has done or shown - showed a lot of options for how models can stay up to date with concept drift in very specific situations
%  motivation - in order to bridge the gap between research and industry, the applicability of approaches need to be studied
%  your research and/or analytical methods - literature based
%  your main findings, results, or arguments - that neural networks can be kept up to date with highly voluminous data using explicit concept drift detection, model approximation / parallelization and batch size increasing
%  implication - what needs to be taken into account when choosing the optimal updating workflow
