\documentclass{article}
\usepackage[utf8]{inputenc}
\usepackage[style=numeric,bibstyle=numeric,backend=biber,natbib=true,maxbibnames=99,giveninits=true,uniquename=init]{biblatex}
\addbibresource{bibliography.bib}

\begin{document}

% this document contains the shortcut names of each paper found, divided by thesis part

% ! to note: I should probably look at publication dates because if something says its the best newest innovation and is 5yrs old it will just mislead

% to add: relevance
\chapter{surveys and other fits-many-section's}

\begin{itemize}
    \item \cite{iotsystems} iotsystems: survey on iot systems
\end{itemize}

\chapter{context}

reference architectures etc.

\begin{itemize}
    \item \cite{dataflow} dataflow: a viewpoint on data
    \item \cite{lambdakappa} lambdakappa: introduction to lambda and kappa architectures
    \item \cite{thelambdarant} thelambdarant: magazine article on lambda and kappa, critical
\end{itemize}

apache tools

\begin{itemize}
    \item \cite{apachebenchmarkI} apachebenchmarkI: a tool benchmarking comparison
    \item \cite{apachebenchmarkII} apachebenchmarkII: another tool benchmarking comparison
    \item \cite{apachesurvey} apachesurvey: a comparison of apache tools for smart cities
    \item \cite{mapreduce} mapreduce: what is mapreduce
\end{itemize}

maritime case

\begin{itemize}
    \item \cite{maritimeinformatics} maritimeinformatics: the introduction to maritime book
     \item \cite{i4sea} i4sea: uprc, system for maritime big data
     \item \cite{uprctrajectorysystem} uprctrajectorysystem
     \item \cite{D1.1} \cite{D4.1} delivarables
     \item \cite{maritimetradvsbigdata} maritimetradvsbigdata: a comparison on traditional vs big data approach for maritime anomaly detection
\end{itemize}

the largest big data systems (these will probably not be needed):

\begin{itemize}
    \item \cite{storm@twitter} storm@twitter: twitter storm, master-worker example
    \item \cite{facebook} facebooks' master-worker-model based systems
    \item \cite{uber} the uber system description
    \item \cite{millwheel} google millwheel, this is often compared to in papers 
\end{itemize} 


\chapter{on iot/ml+iot/smart city systems}

\begin{itemize}
    \item \cite{edgelatency} edgelatency: empirical comparison on end-to-end times with centralized vs edge computing (data amount is small however)
    \item \cite{geolytics} geolytics: multi-location supporting iot system
    \item \cite{edgeiot} edgeiot: iot edge computing
    \item \cite{mliot} mliot: a proposal for ml+iot optimized system
    \item \cite{gpuinsmartcity} gpuinsmartcity
    \item \cite{e2eiotstack} e2eiotstack: bluiding blocks into a full system
    \item \cite{edgefogcloud} edgefogcloud: the main reference on iot systems from D1.1, given as backing to the argument that a hybrid deployment scheme with edge fog and cloud should be used
    \item \cite{fogsurvey} fogsurvey: a survey on fog computing for IoT
\end{itemize}

other similar to VesselAI system descriptions

\begin{itemize}
    \item \cite{anomalysystem}: a ''lightning-fast'' system for maritime anomaly detection
    \item \cite{timecriticalmarineaero}: a system for time-critical maritime and aerial applications
    \item \cite{multilambdaforlml} multilambdaforlml: multiagent lambda architecture for lml
\end{itemize}

on lifelong learning
\begin{itemize}
    \item \cite{lmlinneuralnets} lmlinneuralnets: overview in implemetations of lifelong learning into neural nets
    \item \cite{lmlsystemframework} lmlsystemframework: framework for managing knowledge adaptation/preservation in lml
    \item \cite{lmlsystems} lmlsystems: an overview on lml & system demands from it
\end{itemize}

todo: find references & add here on things that are model lifetime extending but not lifelong learning

not in .bib:

the ferari project: http://www.ferari-project.eu/objectives/ aims to create a distributed sensor system, ml integrated
-> find relevant papers/delivarables from there

potential to-adds:

Detecting Representative Trajectories from Global AIS Datasets in IEEE, a vesselai publication


\printbibliography

\end{document}
