\documentclass{article}
\usepackage[utf8]{inputenc}
\usepackage[style=numeric,bibstyle=numeric,backend=biber,natbib=true,maxbibnames=99,giveninits=true,uniquename=init]{biblatex}
\addbibresource{bibliography.bib}

\begin{document}

% this document contains the shortcut names of each paper found, divided by thesis part

% ! to note: I should probably look at publication dates because if something says its the best newest innovation and is 5yrs old it will just mislead

% to add: relevance
\chapter{surveys and other fits-many-section's}

\begin{itemize}
    \item \cite{iotsurvey} iotsurvey: general survey and taxonomy on iot
    \item \cite{iotsystems} iotsystems: survey on iot systems
\end{itemize}

\chapter{context}

often used architectures etc

\begin{itemize}
    \item \cite{dataflow} dataflow: a viewpoint on data
    \item \cite{lambdakappa} lambdakappa: introduction to lambda and kappa architectures
    \item \cite{mapreduce} mapreduce: what is mapreduce
    \item \cite{apachesurvey} apachesurvey: a comparison of apache tools for smart cities
\end{itemize}

maritime case

\begin{itemize}
    \item \cite{maritimeinformatics} maritimeinformatics: the introduction to maritime book
    \item \cite{weatherroutingspeedoptimum} weatherroutingspeedoptimum: from napa, labeled as central
    \item \cite{weatherroutinguncertaintyI} weatherroutinguncertaintyI: on uncertainty in weather routing, probably less relevant
     \item \cite{weatherroutinguncertaintyII} weatherroutinguncertaintyII: another one on uncertainty in weather routing, probably less relevant
     \item \cite{i4sea} i4sea: uprc, system for maritime big data
     \item \cite{uprctrajectorysystem} uprctrajectorysystem
\end{itemize}

the largest big data systems (these will probably not be needed):

\begin{itemize}
    \item \cite{storm@twitter} storm@twitter: twitter storm, master-worker example
    \item \cite{facebook} facebooks' master-worker-model based systems
    \item \cite{uber} the uber system description
    \item \cite{millwheel} google millwheel, this is often compared to in papers 
\end{itemize} 


\chapter{vesselai-like systems}

\begin{itemize}
    \item \cite{edgelatency} edgelatency: empirical comparison on end-to-end times with centralized vs edge computing (data amount is small however)
    \item \cite{geolytics} geolytics: multi-location supporting iot system
    \item \cite{edgeiot} edgeiot: iot edge computing
    \item \cite{mliot} mliot: a proposal for ml+iot optimized system
    \item \cite{gpuinsmartcity} gpuinsmartcity
    \item \cite{e2eiotstack} e2eiotstack: bluiding blocks into a full system
\end{itemize}

not in .bib:

deliverables from vesselai: d1.2(user stories), d1.1. (initial requirements) how to reference these?

the ferari project: http://www.ferari-project.eu/objectives/ aims to create a distributed sensor system, ml integrated
-> find relevant papers/delivarables from there

potential to-adds:

Detecting Representative Trajectories from Global AIS Datasets in IEEE, a vesselai publication

check most interesting references from the list above! esp. d 1.1 relevant references list!

\printbibliography

\end{document}
