SECT 2.2: systems intro

Best practices in big data systems design: organization and principles
ingressi: miksi tää on tässä, mitä tullaan kertoo
luettele bd systemin yleiset osat: data ingestion, prep, storage, trianing, registry, inference

esittele bd systemin yleiset osat osa kerrallaan

	model development workflow is omitted, assumption: model is developed. (development lifecycle is nontrivial for organization but is out of scope) assumption: model had ready is a neural net, affecs training
	DATA INGESTION
	description of taking in sensor data
	batch and stream: what are they (batch as collected chunk of stream)
	the general schemes: lambda and kappa (shortly)

	DATA PREPARATION
	what it is: existing text is good
	include: data needs to be validated (from mlops guide)

	DATA STORAGE
	common ways of storing
	note on what is able to store: 
		not like dump all there and use when you have time
		memory is very scarce: only a few days data can be saved, one-pass data is common practice in streaming
		
	MODEL TRAINING
	intro to federated learning
	(eg DATE n kung fu papers (1st & 2nd from lucy) state that in ML systems distribution is of high importance)
	
	MODEL EVALUATION & VALIDATION
	here models are made sure that they work. like model testing phase

	MODEL REGISTRY
	quick naming: the storage of trained models, memory-wise small

	MODEL INFERENCE
	mention the fact that many devices are possible, this is big iot research field

näytä kuva: the components are summarized in Figure...

kerro asiat joita käsitellään: trigger = under which conditions retraining happens
training: the actual retraining

näytä kuva: tää kuva visualisoi missä osassa järjestelmää ollaan (google mlops se toinen kuva)

kerro bestpracticeist: simple mature automated (automl mlops)

sect conclusion