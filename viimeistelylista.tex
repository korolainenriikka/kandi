KANDI VIIMEISTELYHOMMAT!!!

viimeistelykierrokset
* acronym-kierros
* oikolukukierros [3]
* viitteet järkeviä -kierros [1]
* tuun kertoo tää kerrottiin -kierros
* voiko tekstiä yksinkertaistaa -kierros [6]

yksittäiset korjaukset - tekniset
* distributionanalyysin alku: sano et analysoidaan arvioidaan distribuutiota tekee tän selväks et tää on mun alote, ei pelkkä täytelause
* maritime yksinään puhekielinen
* tarkista onko keywordit edelleen relevantit (mlops vois lisää?)
* ensimmäinen tutkimuskysymyksen muotoilu on liian abstrakti (?)
* kuvien resoluutiot kuntoon
* merkkaa supervisors ja examiners
* extreme scale data kursiivi ei oo konsistentti muun tyylin kans. norm tai bold.
* 3.2 lopussa aik tökkö learning task learning task.
* leskirivit s 12: enlargethispage, 6mm tai 7mm antaa yhden rivin lisää. s 14 kans. enlargethispage negatiivisella arvolla.
* vaihda also known as lause abstraktista, aka on puhekielisyydelle varattu
* vaihda secondary question on esim followup question tai jotai 2nd research quussiin on more detailed concrete level broken out to several questions
* muotoile tutkimuskysymys niin, että given on päälauseessa

sisällölliset korjaukset
* tee tarkemmin selväks mikä on vesselain ja syväoppimisen välinen linkki
* tulostiivistelmä abstraktiin
* maintenance ilmaisutyyli enemmän johonkin sanaan joka viittaa pidä samana tai paranna
* onko efficient data management ihan oikee sanavalinta?
* monitoroinnist puhuminen drift detection sectionissa
* tarkista introssa et tutkitut research fieldit on nimetty oikein
* secondary research questionit: vaihda sellaisiin joihin oikeasti vastataan (automationiin kyl!)
* uudelleenkirjota intron lopun thesis is organized as follows
* mieti toi karakterisoiko adaptive learningin käsite oikeest tätä kandii
* käytä continual learning ja extreme scale dataa oikeesti, ne on selitetty.
* starting point on small sample of data, et tää lukee kaikkial samal taval
* CONTEXT!!! [2]
* rewrite ylipäätään kaikkiin luku3 osuuksiin + conclusioniin [5]
* onko 3.2 tutkimuskysymys oikeesti generalization?
* tarkista conclusionin osuus siitä mikä vaikutti mihinkin johtopäätökseen.
* tarkista onko otsikko vielä relevantti

ihan viimeset korjaukset
* pdf tiedostonimi sukunimi etunimi tutkielman otsikko väleillä erotettuna.pdf
* leskirivien taitto: yksittäinen rivi ei yksin sivulla, kappaleet mahd mukaan kokonaisia

[1] (liikaa viitteitä on läpikiirehtimisjuttu!)
* kaikki semmone mikä tarvii viitteen sil on viite [4]
* itsestäänselvyyksille ei oo viitteitä
* viitteen tyyppi on oikea: tutkimustuloksissa ei viitata surveyyn, survey vaan määritelmille ja yleisille lausunnoille
* viittaamiset on silleen et siin paperissa ne sanoo sen asian eikä sitä asiaa oo haettu suoraan jostain muusta paperista
* joka viitteelle on syy (lambdakappan pärjää vähemmil viitteil?) (ja ainaki eddm ddm fddm!)
(heikko syy paperille olla lähdeluettelossa: lähteen käytöllä on hyvin köykäiset syyt, asia yleisesti tunnettu tai on mainittuna jo jossain toisessa lähteessä.)

[2]
* kuvalla voi kontekstualisoida?

[3]
retraining vai re-training?
preprocessing vai pre-processing? tähän konsistenttiutta

[4] asioita jotka etsii viitettä
* data is only stored for a few days max
* one-pass processing data is not stored at all -> bifet
* mlflow
* only a few recent works address automl concept drift

[5]
* 3.1: data quality assurance on aika irrallisia pointteja, koheesiota tarvitaan
* data error käsitteenä vois nimee, määrittää ja käyttää. selkiyttäis

[6]
* p 11 additional entropy to data lause on liian pitkä
* mlflowiin loppuva lause on liian pitkä

LEALLA TARKASTUTETTAVAA VIELÄ
lähdeluettelo viimeset fixit ku arxivei on vaihdettu pois